\documentclass{article}
\usepackage[utf8]{inputenc}
\usepackage{textcomp}
\usepackage{graphicx}
\usepackage{float}
\usepackage{array}
\usepackage{amsmath} % Math aligning equation
\usepackage{verbatim} % Command \verb
\usepackage{titlesec} % Righe di separazione
\usepackage{subcaption}
\usepackage[table]{xcolor}
\usepackage{tikz}
\usepackage{float}



% Tabelle
\usepackage{tabu}
\usepackage{caption} 
\captionsetup[table]{skip=2pt}

% Impostazioni di pagina e margini
\usepackage[a4paper, margin=2.54cm]{geometry}

% Spacing nelle liste
\usepackage{enumitem}
\setlist{topsep=2pt, itemsep=2pt, partopsep=2pt, parsep=2pt}

% Cambio di nome di contenuti Latex
\renewcommand*\contentsname{Indice}
\renewcommand{\figurename}{Figura}
\renewcommand{\tablename}{Tabella}

% Checkmarks
\usepackage{pifont}
\newcommand{\cmark}{\ding{51}} % V
\newcommand{\xmark}{\ding{55}} % X

% Header & Footer
\usepackage{fancyhdr}
\pagestyle{fancy}
\fancyhf{}
\lhead{Progetto finale di Reti Logiche - a.a. 2020/2021}
%\rhead{Elisabetta Fedele}
\cfoot{\thepage}

% Titolo e informazioni
\title{Progetto finale di Reti Logiche}
\author{Elisabetta Fedele - Matricola n. 865833}
\date{Anno Accademico 2018/2019}


\begin{document}

%\maketitle
%========= Title Page
\thispagestyle{empty} 
\begin{titlepage}
    \begin{center}
       \vspace*{2cm}
       {\LARGE Prova Finale (Progetto di Reti Logiche) [051228]} %%Replace this with the Title of your research
       \vspace{1cm}
    \begin{large}   
    
        
        { Prof. Fabio Salice - Anno 2020/2021}
       \vspace{8cm}
        
        {Elisabetta Fedele (Codice Persona 10631762 - Matricola 911320)\\Filippo Lazzati (Codice Persona 10629918 - Matricola 910614)}
       \vspace{3cm}
       
    \end{large}  
   \end{center}
\end{titlepage}
\tableofcontents


%%%%%%%%%%%%%%%%%%%%%%%%%%%%%%%%%%%%%%%%%%%%
%%%%%%%%%%%%%%% INTRODUZIONE %%%%%%%%%%%%%%%
%%%%%%%%%%%%%%%%%%%%%%%%%%%%%%%%%%%%%%%%%%%%
\pagebreak
\section{Introduzione} \label{subsection-introduz}

\subsection{Scopo del progetto}
Sia data un’immagine 8 bit gray-scale di dimensione massima pari a 128 x 128 pixel. 
\\L’obiettivo del progetto è l’implementazione di un componente hardware descritto in VHDL che, ricevuta in ingresso tale immagine e dei segnali di controllo, sia in grado di applicarvi l’algoritmo di equalizzazione fornito nelle specifiche e restituire in uscita i risultati.

\subsection{Algoritmo di equalizzazione} \label{subsection-ob-vel}
Equalizzare l’istogramma di un’immagine significa applicare una funzione di trasformazione ai pixel in essa contenuti con lo scopo di aumentare il contrasto globale.
L’algoritmo utilizzato, una versione semplificata del più complesso algoritmo standard di equalizzazione, si articola nei seguenti passi:
\vspace{3.5pt}
\begin{enumerate}
\item Ricerca del minimo valore (\verb^MIN_PIXEL_VALUE^) e del massimo valore (\verb^MAX_PIXEL_VALUE^) 
\item Calcolo di delta value 
\\$\verb^DELTA_VALUE^ = \verb^MAX_PIXEL_VALUE^ - \verb^MIN_PIXEL_VALUE^$
\item Calcolo del valore di shift
\\$\verb^SHIFT_LEVEL^ = 8 - \lfloor{\log_2 (\verb^DELTA_VALUE^ +1)}\rfloor $
\item Trasformazione di ogni pixel
\\$\verb^TEMP_PIXEL^ = (\verb^CURRENT_PIXEL_VALUE^ - \verb^MIN_PIXEL_VALUE^) << \verb^SHIFT_LEVEL^$
\\$\verb^NEW_PIXEL_VALUE^ = min (255, \verb^TEMP_PIXEL^)$
\end{enumerate}
\vspace{3.5pt}
Si consideri, per esempio, l' immagine e il relativo istogramma in Figura 1:
\begin{figure}[h]
\begin{subfigure}{0.4\textwidth}
\includegraphics[width=0.7\linewidth]{Figures/duomo_pre.jpeg} 
%\caption{Caption1}
\label{fig:subim1}
\end{subfigure}
\begin{subfigure}{0.6\textwidth}
\includegraphics[width=0.6\linewidth]{Figures/isto_duomo_pre.png}
%\caption{Caption 2}
\label{fig:subim2}
\end{subfigure}
\caption{Immagine originale e relativo istogramma}
\label{fig:image2}
\end{figure}
\vspace{3.5pt}
\\Nell’istogramma si nota come la distribuzione dei grigi nell’immagine non è uniforme, ma è spostata verso i valori alti. Questo lo si può anche comprendere osservando l'immagine, che risulta sbiadita. 
\\Il risultato dell'applicazione dell'algoritmo è mostrato in Figura 2, dove il nuovo istogramma mostra come i valori di intensità dei singoli pixel sono stati ridistribuiti in modo da coprire l'intero range proposto.
%\newpage
\begin{figure}[h]
\begin{subfigure}{0.4\textwidth}
\includegraphics[width=0.7\linewidth]{Figures/duomo_post.jpg} 
%\caption{Caption1}
\label{fig:subim1}
\end{subfigure}
\begin{subfigure}{0.6\textwidth}
\includegraphics[width=0.6\linewidth]{Figures/isto_duomo_post.png}
%\caption{Caption 2}
\label{fig:subim2}
\end{subfigure}
\caption{Immagine trasformata e relativo istogramma}
\label{fig:image2}
\end{figure}
%\vspace{3.5pt}
\newpage

\subsection{Interfaccia componente}
Il componente presenta la seguente interfaccia:
\begin{figure}[h]
\includegraphics[width=0.7\linewidth]{Figures/entity.png} 
%\caption{}
\label{fig:subim1}

\end{figure}
\\Nello specifico:
\begin{itemize}
    \item \verb^i_clk^ è il segnale di \verb^CLOCK^ in ingresso;
    \item \verb^i_rst^ è è il segnale di \verb^RESET^ che inizializza la macchina;
    \item \verb^i_start^ è il segnale di \verb^START^;
    \item \verb^i_data^ è il segnale che arriva dalla memoria in seguito ad una richiesta di lettura;
    \item \verb^o_address^ è il segnale di uscita che manda l’indirizzo alla memoria;
    \item \verb^o_done^ è il segnale di fine elaborazione;
    \item \verb^o_en^ è il segnale di \verb^ENABLE^ da inviare alla memoria per poter comunicare sia in lettura che in scrittura;
    \item \verb^o_we^ è il segnale di \verb^WRITE ENABLE^ per abilitare la scrittura in memoria;
    \item \verb^o_data^ è il segnale di uscita dal componente verso la memoria.
\end{itemize}

\newpage
\subsection{Dati e descrizione memoria}
I dati, ciascuno di dimensione 8 bit, sono memorizzati in una memoria con indirizzamento al byte nel seguente modo:
\renewcommand{\arraystretch}{2}
\definecolor{Gray}{gray}{0.95}
\newcolumntype{g}{>{\columncolor{Gray}}m{15em}}
\begin{center}
\centering
\begin{tabular}{l c}
&
\hspace{-2.5cm}
\begin{tabular}[c]{ | g | c  l } 
\cline{0-0}
Numero colonne (C) & & Indirizzo 0 \\ 
\cline{0-0}
Numero righe (R) & & Indirizzo 1\\ 
\cline{0-0}
\end{tabular}\\
Pixel immagine originale & 
\hspace{-1.6cm}
$\left\{
\begin{tabular}[c]{ | g | c  l } 
Pixel 1  & & Indirizzo 2\\ 
\cline{0-0}
Pixel 2 \\
\cline{0-0}
...\\ 
\cline{0-0}
Pixel CxR  & & Indirizzo 1 + CxR\\
\cline{0-0}
\end{tabular}
\right.\kern-\nulldelimiterspace$ \\
Pixel immagine originale &
\hspace{-0.4cm}
$\left\{
\begin{tabular}[c]{ | g | c  l } 
Pixel 1  & & Indirizzo 2 + CxR\\ 
\cline{0-0}
Pixel 2  & & Indirizzo 3 + CxR\\ 
\cline{0-0}
...\\ 
\cline{0-0}
Pixel CxR & & Indirizzo 2 + CxR + CxR\\
\cline{0-0}
\end{tabular}
\right.\kern-\nulldelimiterspace$
\end{tabular}
\end{center}
Nei primi due byte sono presenti rispettivamente il numero di colonne e righe dell'immagine da analizzare.\\
I vari pixel, ciascuno di 8 bit, sono memorizzati in memoria a partire dalla posizione 2.\\
I pixel dell' immagine equalizzata, invece, vengono inseriti in memoria a partire dalla posizione 2+(C*R)


\vspace{4mm}
%\titlerule[0.4pt]
\subsection{Protocollo di inizio e fine computazione}
Quando il segnale di ingresso \verb^i_start^ viene portato a 1, il componente progettato inizia la computazione spostandosi nello stato \verb^WAIT^ in cui inizia a richiedere dati in memoria. Il segnale \verb^i_start^ rimmarrà alto fino a quando \verb^o_done^ non verrà portato alto. 
\\Dopo aver scritto l'intero risultato in memoria, il componente alza a 1 il segnale di \verb^o_done^, per segnalare la fine dell'elaborazione. Il segnale \verb^o_done^ rimane alto fino a quando il segnale di \verb^i_start^ non viene riportato a 0; dopodichè anche \verb^o_done^ viene abbassato. Nell'intervallo di tempo durante cui \verb^i_start^ = 0 e \verb^o_done^ = 1 non può essere dato nessun nuovo segnale di \verb^START^.

\subsection{Protocollo di accesso alla memoria}
Durante la computazione posso accedere in memoria all'indirizzo indicato in \verb^o_address^ sia in lettura che in scrittura. In particolare:
\begin{itemize}
    \item Se \verb^o_en^=1 e \verb^o_we^=0 leggerò dalla memoria il dato contenuto nel segnale \verb^i_data^
    \item Se \verb^o_en^=1 e \verb^o_we^=1 scriverò in memoria il dato contenuto nel segnale \verb^o_data^
\end{itemize}




\newpage





















%%%%%%%%%%%%%%%%%%%%%%%%%%%%%%%%%%%%%%%%%%%%
%%%%%%%%%%%%%%% ARCHITETTURA %%%%%%%%%%%%%%%
%%%%%%%%%%%%%%%%%%%%%%%%%%%%%%%%%%%%%%%%%%%%
\pagebreak
\section{Architettura}
Per la progettazione del componente è stata scelta l'implementazione tramite una macchina a stati finiti (FSM) di Mealy non completamente specificata. Infatti il valore delle uscite è funzione degli ingressi e per alcune transizioni è indifferente (DC).


\subsection{FSM}
%descrizone stati + disegnp
La macchina è composta da 10 stati. Qui di seguito è fornita una breve descrizione qualitativa del loro funzionamento.
%\vspace{-0.55cm} 
\subsubsection{\textbf{IDLE} state}
%\vspace{-0.6cm} 
Stato iniziale e di default della macchina, in cui si attende il segnale di \verb^i_start^ e in cui si torna in caso di ricezione di un segnale di reset
%\vspace{-0.55cm}
\subsubsection{\textbf{WAIT} state}
%\vspace{-0.6cm} 
Stato in cui si attende la risposta della memoria in seguito alla richiesta di un dato
%\vspace{-0.55cm}
\subsubsection{\textbf{COL} state}
%\vspace{-0.6cm} 
Stato in cui si legge la prima cella di memoria, contenente il numero di colonne dell'immagine da trasformare
%\vspace{-0.55cm}
\subsubsection{\textbf{ROW} state}
%\vspace{-0.6cm} 
Stato in cui si legge la seconda cella di memoria, contenente il numero di righe dell'immagine da trasformare e si calcola il valore dell'indirizzo di memoria che contiene il valore dell'ultimo pixel. Tale valore viene memorizzato nel signal \verb^max_address^.
%\vspace{-0.55cm}
\subsubsection{\textbf{MIN\textunderscore MAX} state}
%\vspace{-0.6cm} 
Stato in cui viene calcolato il valore del minimo e del massimo pixel presente nell'immagine fornita. Ciò avviene scorrendo la memoria dall'indirizzo 2 all'indirizzo \verb^max_address^ calcolato in \verb^ROW^.\\Il valore del minimo e del massimo sono poi memorizzati rispettivamente nei signals \verb^min^ e \verb^max^).
%\vspace{-0.55cm}
\subsubsection{\textbf{ARG} state}
%\vspace{-0.6cm} 
Stato in cui viene calcolato il valore di max-min+1 e viene salvato nel signal delta\textunderscore value.
%\vspace{-0.55cm}
\subsubsection{\textbf{LOG} state}
%\vspace{-0.6cm} 
Stato in cui viene calcolato lo shift level il cui valore viene salvato nel signal shift\textunderscore value.
%\vspace{-0.55cm}
\subsubsection{\textbf{UPDATE\textunderscore R} state}
%\vspace{-0.6cm} 
Stato in cui vengono letti uno alla volta i valori originali dei pixel, che vengono poi modificati e salvati nella variabile temp\textunderscore pixel\textunderscore shifted in attesa di essere salvati in memoria da UPDATE\textunderscore W.
%\vspace{-0.55cm}
\subsubsection{\textbf{UPDATE\textunderscore W} state}
%\vspace{-0.6cm}
Stato in cui vengono salvati in memoria (uno per volta) i valori dei pixel dell'immagine trasformata. Al termine della computazione, prima che avvenga la transizione verso DONE, setta tutti i segnali e le uscite ai valori di default.
%\vspace{-0.55cm}
\subsubsection{\textbf{DONE state}}
%\vspace{-0.6cm}
Stato per la terminazione di un’istanza di computazione che riporta allo stato di IDLE per rendere il modulo disponibile a ricevere nuove immagini.

\newpage
\begin{figure}[H] % ’ht’ tells LaTeX to place the figure ’here’ or at the top of the page
    \centering % centers the figure
    \begin{tikzpicture}[->,
state/.style={
draw,
fill=white,
circle, 
minimum width= {width("UPDATE")},
font=\tiny}]
    \node[state] (q1) {\tiny IDLE};
    \node[state, below of=q1, yshift=-1.5cm] (q2) {\tiny COL};
    \node[state, below of=q2, yshift=-1.5cm] (q3) {\tiny ROW};
    \node[state, below of=q3, yshift=-1.5cm] (q4) {\tiny MIN\textunderscore MAX};
    \node[state, below of=q4, yshift=-1.5cm] (q5) {\tiny ARG};
    \node[state, below of=q5, yshift=-1.5cm] (q6) {\tiny LOG};
    \node[state, below of=q6, yshift=-1.5cm] (q7) {\tiny UPDATE\textunderscore R};
    \node[state, below of=q7, yshift=-1.5cm] (q8) {\tiny UPDATE\textunderscore W};
    \node[state, below of=q8, yshift=-1.5cm] (q9) {\tiny DONE};
    \node[state, right of=q3, xshift=9cm] (q10) {\tiny WAIT};
    \draw   (q1) edge[bend left, below] node{} (q10)
            (q2) edge[below] node{} (q10)
            (q10) edge[bend right, above] node [above, pos=0.7]{\tiny state\textunderscore return = COL} (q2) 
            (q3) edge[bend right=10, below] node{} (q10)
            (q10) edge[above] node[sloped, above,pos=0.5]{\tiny state\textunderscore return = ROW} (q3)
            (q10) edge[below] node[sloped, above,pos=0.75]{\tiny state\textunderscore return = MIN\textunderscore MAX} (q4)
            (q4) edge[bend right=10,below] node[below,pos=0.2]{\tiny state\textunderscore count $\neq$ max\textunderscore address+1} (q10)
            (q4) edge[below] node[right]{\tiny state\textunderscore count = max\textunderscore address+1} (q5)
            (q5) edge[below] node{} (q6)
            (q6) edge[above] node{} (q10)
            (q10) edge[below] node[sloped, below,pos=0.5]{\tiny state\textunderscore return = UPDATE\textunderscore READ} (q7)
            (q7) edge[above] node{} (q8)
            (q8) edge[bend right, above] node[sloped, below,pos=0.5]{\tiny count $\neq$ max\textunderscore address-1} (q10)
            (q8) edge[above] node[right,pos=0.5]{\tiny count = max\textunderscore address-1} (q9)
            (q9) edge[bend left,above] node{} (q1);
\end{tikzpicture}
    \caption{FSM}
    \label{fig:my_label}
\end{figure}
\newpage
\subsection{Uscite di default}
Determinati segnali di uscita non sono rilevanti in determinati casi. In particolare:
\begin{itemize}
    \item se \verb^i_start^ = '0': tutte le uscite del componente non sono rilevanti
    \item se \verb^o_en^ = '1' e \verb^o_we^ = '0': l'uscita \verb'o_data' non è rilevante
\end{itemize}
Quando le uscite non sono rilevanti assumono un valore di default. \\Di seguito riportiamo i valori di default assegnati alle uscite e le motivazioni delle nostre scelte. 
\begin{table}[H] % Nota viene usato: % 
    \centering
    \begin{tabu*} to 1.0\textwidth { |X[0.7c]|X[0.7c]|X[2.4c]|}
        \hline
        \textbf{Nome} & \textbf{Default} & \textbf{Motivazione}\\
        \hline
        \verb^o_address^ & others \Rightarrow ’0’ & L'indirizzo 0 è il primo indirizzo di lettura\\
        \hline
        \verb^o_en^ & 0 & Fino a quando \verb^i_start^ = '0' non accedo alla memoria\\
        \hline
        \verb^o_we^ & 0 & Il primo accesso alla memoria è in lettura\\
        \hline
        \verb^o_data^ & others \Rightarrow ’0’ & Scelta arbitraria\\
        \hline
        \verb^o_done^ & others \Rightarrow ’0’ & All'inizio dell'elaborazione deve valere 0\\
        \hline
    \end{tabu*}
\end{table}




\subsection{Variabili, signals e valori di default}
Di seguito riportiamo i signals e le variabili utilizzate e i rispettivi valori di default a esse assegnati. 
\begin{table}[H] % Nota viene usato: % 
    \centering
    \begin{tabu*} to 1.0\textwidth { |X[1.2c]|X[2.4c]|X[0.9c]|X[2.4c]|}
        \hline
        \textbf{Nome} & \textbf{Uso} & \textbf{Default} & \textbf{Motivazione}\\
        \hline
        \verb^min^ & Contiene minimo valore & 255 & Qualunque valore è \leq 255\\
        \hline
        \verb^max^ & Contiene massimo valore & 0 & Qualunque valore è \geq 0\\
        \hline
        \verb^count^ & Helper per leggere i pixel & 2 & I pixel iniziano dall'indirizzo 2\\
        \hline
        \verb^max_address^ & Contiene indirizzo fine lettura & 2 & I pixel iniziano dall'indirizzo 2\\
        \hline
        \verb^delta_value^ & Contiene \verb^DELTA_VALUE^ + 1 & others \Rightarrow ’0’ & Scelta arbitraria\\
        \hline
        \verb^shift_level^ & Contiene \verb^SHIFT_LEVEL^ & 0 & 0 \Rightarrow istogramma già equalizzato\\
        \hline
        \verb^state_return^ & Contiene il \verb^next_state^ di \verb^WAIT^ & \verb^IDLE^ & Scelta arbitraria\\
        \hline
        \verb^current_state^ & Contiene lo stato attuale & \verb^IDLE^ & Scelta arbitraria\\
        \hline
        \verb^col_count^ & Memorizza numero colonne & 0 & Scelta arbitraria\\
        \hline
        \verb^o_address_reg^ & Lettura dell’uscita \verb^o_address^ & others \Rightarrow ’0’ & Scelta arbitraria\\
        \hline
    \end{tabu*}
\end{table}


\newpage
\subsection{Scelte implementative}
%processi 
Nella realizzazione del componente come FSM si è deciso di utilizzare due soli processi: \verb^state_reg^ e \verb^delta_lambda^.
\begin{itemize}
    \item Il process \verb^state_reg^ definisce dei registri parallelo-parallelo dotati di reset asincrono, che vengono scritti e letti in corrispondenza dei fronti di salita del clock (\verb^i_clk^), aggiornando rispettivamente il valore dello stato e dell’uscita. Si è deciso di realizzare un unico process unendo il registro di memorizzazione degli stati con quello delle uscite per una questione di leggibilità e sintesi, senza alcun vantaggio dal punto di vista dell’implementazione hardware.
    \item Il process \verb^delta_lambda^ rappresenta la FSM e, analizzando lo stato presente e gli ingressi, determina lo stato successivo e le uscite
\end{itemize}




\subsection{Scelte di progettazione}
Per l’implementazione del componente si utilizza la seguente FPGA: xc7a200tfbg484-1, appartenente alla famiglia Artix 7 prodotta dall’azienda Xilinx. Ha un’altezza di 1mm e occupa una superficie di 23x23mm. A seconda della versione cambia il range di temperature di lavoro possibili. Il suo speed range è -1, dunque la sua frequenza di lavoro interna massima è inferiore ai valori delle corrispondenti frequenze delle altre FPGA appartenenti alla medesima famiglia Artix 7.







%%%%%%%%%%%%%%%%%%%%%%%%%%%%%%%%%%%%%%%
%%%%%%%%%%%%%%% SINTESI %%%%%%%%%%%%%%%
%%%%%%%%%%%%%%%%%%%%%%%%%%%%%%%%%%%%%%%
\pagebreak
\section{Sintesi}
\subsection{Area occupata}


\subsection{Report di timing} %immagine percorso pessimo


\subsection{Codifica degli stati utilizzata}


\subsection{Warnings post synthesis}
Tutti i warning generati dal tool di sintesi durante lo sviluppo sono stati risolti. Tra questi anche i più "comuni" come i warning per latch inferiti e warning per segnali presenti nel processo ma non inseriti nella sensivity list.\\
Non è quindi presente alcun warning nella versione finale del componente.

\subsection{Note aggiuntive sulla sintesi}
Si è utilizzato il tool "Vivado 2020.2 WebPACK Edition" impostato con i parametri di default.

\vspace{4mm}
\titlerule[0.4pt]


%%%%%%%%%%%%%%%%%%%%%%%%%%%%%%%%%%%%%%%%%%%
%%%%%%%%%%%%%%% SIMULAZIONI %%%%%%%%%%%%%%%
%%%%%%%%%%%%%%%%%%%%%%%%%%%%%%%%%%%%%%%%%%%
\pagebreak
\section{Simulazioni}

\subsection{Casi limite}
In questo test bench viene provato un caso normale, senza casi limite, però ci permette di vedere alcune delle ottimizzazioni all'opera. Vediamo perciò in dettaglio il funzionamento del componente con questo test bench.

\subsubsection{Dati del test bench}
\begin{table}[H]
    \begin{tabu*} to 0.8\textwidth { X[1.7l] X[1.1c] X[0.7c] X[1.1c]}
        \textbf{Contenuto} & \textbf{Valore} & \textbf{Indirizzo} & \textbf{Da considerare} \\
         Maschera d'ingresso & 1011'1001 (185) & 0 & - \\
         X centroide 1 & 75 & 1 & \cmark \\
         Y centroide 1 & 32 & 2 & \cmark \\
         X centroide 2 & 111 & 3 & \xmark \\
         Y centroide 2 & 213 & 4 & \xmark \\
         X centroide 3 & 79 & 5 & \xmark \\
         Y centroide 3 & 33 & 6 & \xmark \\
         X centroide 4 & 1 & 7 & \cmark \\
         Y centroide 4 & 33 & 8 & \cmark \\
         X centroide 5 & 80 & 9 & \cmark \\
         Y centroide 5 & 35 & 10 & \cmark \\
         X centroide 6 & 12 & 11 & \cmark \\
         Y centroide 6 & 254 & 12 & \cmark \\
         X centroide 7 & 215 & 13 & \xmark \\
         Y centroide 7 & 78 & 14 & \xmark \\
         X centroide 8 & 211 & 15 & \cmark \\
         Y centroide 8 & 121 & 16 & \cmark \\
         X del punto da valutare & 78 & 17 & - \\
         Y del punto da valutare & 33 & 18 & - \\
    \end{tabu*}
\end{table}

\subsubsection{Elaborazione}
\begin{figure}[H]
    \centering
    \caption{Test bench 0, waveform dei segnali in Behavioral Simulation}
    \includegraphics[width=1.0\textwidth]{images/test-bench-0.png}
    \label{fig:test-bench-0}
\end{figure}
Come si può vedere in figura \ref{fig:test-bench-0}, il componente impiega soltanto 11 cicli di clock dalla ricezione del segnale di start fino alla segnalazione di fine con \verb^o_done^.\\
Vediamo cosa succede in questi 11 cicli di clock con cui il componente arriva al risultato finale:
\renewcommand{\labelenumi}{\Roman{enumi}}
\begin{enumerate}
    \item Alla ricezione del segnale di start la FSM del componente passa allo stato \verb^S_START^, qui manda come indirizzo di lettura l'indirizzo 0.
    \item La FSM passa allo stato \verb^S_INPUT_MASK^. Il componente riceve il contenuto della cella di memoria 0: 185. Questa è la maschera d'ingresso che verrà salvata nel relativo registro. Infine manda come indirizzo di lettura l'indirizzo 17.
    \item La FSM passa allo stato \verb^S_COORD_X^. Il componente riceve il contenuto della cella di memoria 17: 78. Questa è la X del punto da valutare che verrà salvata nel relativo registro. Infine manda come indirizzo di lettura l'indirizzo 18.
    \item La FSM passa allo stato \verb^S_COORD_Y^. Il componente riceve il contenuto della cella di memoria 18: 33. Questa è la Y del punto da valutare che verrà salvata nel relativo registro. Infine manda come indirizzo di lettura l'indirizzo 1.
    \item La FSM passa allo stato \verb^S_CX^. Il componente riceve il contenuto della cella di memoria 1: 75. Questa è la X del primo centroide che verrà salvata nel relativo registro. Infine manda come indirizzo di lettura l'indirizzo 2.
    \item La FSM passa allo stato \verb^S_CY^. Il componente riceve il contenuto della cella di memoria 2: 32. Questa è la Y del primo centroide con cui calcola la distanza minima col punto da valutare. Infine manda come indirizzo di lettura l'indirizzo 7, \textbf{vengono quindi saltati i centroidi 2 e 3 siccome non sono da considerare per la maschera}.
    \item La FSM passa allo stato \verb^S_CX^. Il componente riceve il contenuto della cella di memoria 7: 1. \textbf{La distanza sulle ascisse tra questo centroide e il punto da valutare è maggiore della distanza minima. Perciò si passa al centroide successivo}. Manda quindi come indirizzo di lettura l'indirizzo 9.
    \item La FSM rimane nello stato \verb^S_CX^. Il componente riceve il contenuto della cella di memoria 9: 80. Questa è la X del quinto centroide che verrà salvata nel relativo registro. Infine manda come indirizzo di lettura l'indirizzo 10.
    \item La FSM passa allo stato \verb^S_CY^. Il componente riceve il contenuto della cella di memoria 10: 35. Questa è la Y del quinto centroide con cui calcola la distanza dal punto da valutare. Si trova che questa distanza è equivalente a quella minima, viene perciò attivato il quinto bit nella maschera di uscita temporanea. Infine manda come indirizzo di lettura l'indirizzo 11.
    \item La FSM passa allo stato \verb^S_CX^. Il componente riceve il contenuto della cella di memoria 11: 12. \textbf{La distanza sulle ascisse tra questo centroide e il punto da valutare è maggiore della distanza minima. Perciò si passa al centroide successivo}. Manda quindi come indirizzo di lettura l'indirizzo 15, \textbf{viene saltato il centroide 6 disattivato nella maschera}.
    \item La FSM rimane nello stato \verb^S_CX^. Il componente riceve il contenuto della cella di memoria 15: 211. \textbf{Anche qui la distanza sulle ascisse è maggiore della distanza minima}. Abbiamo perciò raggiunto il risultato finale: 0001'0001 (17). Esso viene mandato in scrittura alla RAM nella cella con indirizzo 19.
\end{enumerate}

\subsubsection{Post-Synthesis Timing Simulation}
Vediamo per questo test bench il comportamento del componente una volta sintetizzato.
\begin{figure}[H]
    \centering
    \caption{Test bench 0, waveform dei segnali in Post-Synthesis Timing Simulation}
    \includegraphics[width=1.0\textwidth]{images/test-bench-0-timing.png}
    \label{fig:test-bench-0-timing}
\end{figure}
\noindent I "don't care" spariscono e sono rimpiazzati da valori reali, questi valori sono ininfluenti e sono dati dalle ottimizzazioni fatte dal tool di sintesi. Inoltre sono comparse alcune alee statiche: di tipo 0 sul segnale \verb^o_we^ e di tipo 1 sul segnale \verb^o_en^. Queste alee sono ininfluenti per il risultato in quanto questi segnali sono usati in modo sincrono: conta soltanto il loro valore nell'intorno del fronte di salita del clock.

\subsection{Test Bench 1 (un solo bit attivato nella maschera d'ingresso)}
Con questo test bench si è voluto testare l'ottimizzazione n.2 della sezione \ref{subsection-ob-vel} a pagina \pageref{subsection-ob-vel}. Si è testato in realtà tutti i 9 casi possibili (zero bit attivati, un bit attivato in 8 possibili posizioni) ma vediamo soltanto uno di questi casi in quanto poi gli altri sono equivalenti.

\subsubsection{Dati del test bench}
\begin{table}[H]
    \begin{tabu*} to 0.8\textwidth { X[1.7l] X[1.1c] X[0.7c] X[1.1c]}
        \textbf{Contenuto} & \textbf{Valore} & \textbf{Indirizzo} & \textbf{Da considerare} \\
        Maschera d'ingresso & 1000'0000 (128) & 0 & - \\
        ... & ... & ... & \xmark \\
        X centroide 8 & 25 & 1 & \cmark \\
        Y centroide 8 & 23 & 2 & \cmark \\
        X del punto da valutare & 86 & 17 & - \\
        Y del punto da valutare & 129 & 18 & - \\
    \end{tabu*}
\end{table}

\subsubsection{Elaborazione}
\begin{figure}[H]
    \centering
    \caption{Test bench 1, waveform dei segnali in Behavioral Simulation}
    \includegraphics[width=0.6\textwidth]{images/test-bench-1.png}
    \label{fig:test-bench-1}
\end{figure}
Come si può vedere in figura \ref{fig:test-bench-1}, il componente impiega soltanto 2 cicli di clock dalla ricezione del segnale di start fino alla segnalazione di fine con \verb^o_done^.\\
Vediamo cosa succede in questi 2 cicli di clock con cui il componente arriva al risultato finale:
\renewcommand{\labelenumi}{\Roman{enumi}}
\begin{enumerate}
    \item Alla ricezione del segnale di start la FSM del componente passa allo stato \verb^S_START^ qui manda come indirizzo di lettura l'indirizzo 0.
    \item La FSM passa allo stato \verb^S_INPUT_MASK^. Il componente riceve il contenuto della cella di memoria 0: 128. A questo punto la funzionalità di assegnamento del segnale \verb^is_immediate^ riconosce che la maschera d'ingresso è una potenza di due. Ciò viene riconosciuto se \(128\&(128-1)\) dà come risultato 0, ed effettivamente si trova che 1000'0000 \& 0111'1111 = 0000'0000. Si è quindi raggiunto il risultato finale che non è altro che la maschera d'ingresso stessa: 128.  Esso viene mandato in scrittura alla RAM nella cella con indirizzo 19.
\end{enumerate}

\subsection{Altri test bench}
Sono stati creati altri test bench per testare alcuni casi limite o per controllare la robustezza del componente. Non li vediamo in dettaglio siccome alla fine seguono tutti lo stesso principio di esecuzione.
\begin{itemize}
    \item Test bench in cui la distanza del punto da valutare coi centroidi è molto alta e tale per cui si devono usare 8 + 1 bit. Si è usato quindi dati in cui il punto da valutare ha ascissa e ordinate dal valore molto basso mentre per i centroidi si è usato sia per la X che la Y valori maggiori di 200.
    \item Test bench in cui alcuni centroidi sono a distanza 0 con il punto da valutare.
    \item Test bench in cui tutti i centroidi sono da considerare e hanno la stessa distanza col punto da valutare (maschera di uscita: 1111'1111).
    \item Test bench in cui si controlla il corretto funzionamento del componente a seguito di elaborazioni successive senza segnali di reset.
    \item Test bench in cui si controlla il corretto funzionamento del componente quando, durante una elaborazione, viene mandato un segnale di reset.
    \item Infine sono stati creati una decina di altri test che non vanno a verificare casi particolari ma semplicemente casi normali con dati diversi. Questi test sono stati progettati in modo che ogni bit nella maschera di uscita venga testato almeno una volta.
\end{itemize}
\begin{figure}[H]
    \centering
    \caption{Esempio di test bench effettuato con elaborazioni successive}
    \includegraphics[width=0.95\textwidth]{images/altri-test-bench.png}
    \label{fig:altri-test-bench}
\end{figure}

\vspace{4mm}
\titlerule[0.4pt]


%%%%%%%%%%%%%%%%%%%%%%%%%%%%%%%%%%%%%%%%%%%
%%%%%%%%%%%%%%% CONCLUSIONE %%%%%%%%%%%%%%%
%%%%%%%%%%%%%%%%%%%%%%%%%%%%%%%%%%%%%%%%%%%
\section{Conclusione}
Riassumendo si è creato un design con queste caratteristiche:
\begin{itemize}
    \item Funzionante in pre e post-sintesi.
    \item Ottimizzato in modo che durante l'elaborazione venga sfruttata al massimo la RAM: a ogni ciclo di clock una lettura o una scrittura.
    \item Ottimizzato in modo che ogni lettura della RAM venga eseguita solo se strettamente necessaria (non viene letto ciò che non serve).
    \item Configurabile con le costanti \verb^MEM_BITS^ e \verb^CELL_BITS^ per adattarsi a differenti tipi di RAM, e con il generic \verb^START_ADDRESS^ per eseguire l'elaborazione all'indirizzo desiderato.
    \item Frequenza massima di clock impostabile a 85.6Mhz.
    \item Utilizzo di LUT pari al 0.15\%.
    \item Utilizzo di FF pari al 0.02\%
\end{itemize}
\end{document}
